%\part{Aspectos Gerais}

\chapter[Referencial Teórico]{Referencial Teórico}
	
	Durante esta seção, questões referentes a todo o contexto abordado neste trabalho serão apresentadas e descritas de forma prática para facilitar o total entendimento do tema trabalhado.

\section{A Robótia e a Auto-Localização}

	Grande parte da capacidade do ser humano de se adaptar ao ambiente, sobrevivendo e evoluindo constantemente se dá à utilização, desde os primórdios da humanidade, de ferramentas de auxílio em atividades importantes para o desenvolvimento de uma civilização, ou até mesmo em questões relacionadas à sobrevivência básica, como busca por alimentação e moradia. Devido ao fato do ser humano sempre buscar evolução, as ferramentas utilizadas por nós também possuem uma tendência a serem evoluídas com o tempo.

	Um exemplo simples que retrata a busca por melhoria nos instrumentos de trabalho pode ser observado em trechos descritos por Aristóteles, em meados do século IV a.c., onde o mesmo discute a possibilidade dos instrumentos realizarem suas próprias tarefas, obedecendo ou, até mesmo, antecipando o desejo das pessoas. Aristóteles ainda não sabia, mas já estava descrevendo o futuro de nossas ferramentas, o nascimento da Robótica.

	Durante os séculos seguintes a humanidade questionou o uso da ciência dentro da Indústria, para que a produção de alimentos e utensilios que possam minimizar as dificuldades encontradas durante a evolução da Humanidade possa ser evoluída e melhorada constantemente. Ao final do século XVI, Francis Bacon já discutia a ideia de que a sabedoria devesse ser aplicada na prática, ou seja, a ciência deveria ser utilizada dentro das Indústrias. Bacon ainda afirmava que o Homem possui o dever de se organizar com o objetivo de melhorar e transformar as condições de vida.