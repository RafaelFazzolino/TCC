
\chapter[Suporte Tecnológico]{Suporte Tecnológico}

Nesta seção serão apresentadas ferramentas e tecnologias utilizadas para suportar o desenvolvimento deste projeto. Desde a organização e definição da metodologia de pesquisa, até o desenvolvimento dos projetos pilotos durante o trabalho. Esta seção está dividida em \textit{Projeto} e \textit{Engenharia de Software}.

\section{Engenharia de Software} % (fold)
\label{sec:engenharia_de_software}
	Neste tópico serão apresentadas ferramentas e tecnologias voltadas ao contexto da Engenharia de Software que são utilizadas durante este trabalho, como por exemplo ferramentas para gerência de configuração e versionamento dos artefatos gerados.

	\subsection{GIT} % (fold)
	\label{sub:git}
	
		A ferramenta GIT\footnote{https://git-scm.com/} foi desenvolvida por Linus Torvalds, mesmo criador do Linux, e assim como ele, é \textit{open-source}. Disponibiliza uma eficiente forma de versionamento e gerenciamento de projetos. 
	% subsection git (end)

	\subsection{Github} % (fold)
	\label{sub:github}
		O Github\footnote{https://github.com} é uma ferramenta mais utilizada para hospedagem remota de projetos GIT. A ferramenta contempla uma \textit{Wiki} para documentação do projeto e sistemas de \textit{Issues} e \textit{Milestones} para gerenciamento de atividades.
	% subsection github (end)

	\subsection{Bonita BPMN} % (fold)
	\label{sub:bonita_bpmn}
		Ferramenta para modelagem de processos \textit{BPMN}\footnote{http://www.bpmn.org/}, o Bonita\footnote{http://www.bonitasoft.com/} foi escolhido graças a sua facilidade de utilização e portabilidade para sistema Linux.
	% subsection bizagi_process_modeler (end)

	\subsection{Linux Mint} % (fold)
	\label{sub:linux_mint}
		O sistema operacional utilizado durante este trabalho será o Linux Mint\footnote{https://www.linuxmint.com/}.
	% subsection linux_mint (end)

	\subsection{LaTeX} % (fold)
	\label{sub:latex}
	
	O LaTeX\footnote{https://www.latex-project.org/} é um sistema para criação de documentos utilizando textos \textit{tex}, foi incialmente desenvolvido por Leslie Lamport, na década de 80. O LaTeX oferece diversos comandos avançados para organização de alto nível de documentos, incluindo facilitadores para citações, bibliografias, fórmulas matemáticas, figuras e tabelas.
	% subsection latex (end)

	\subsection{Sublime Text 3} % (fold)
	\label{sub:sublime_text_3}
		O Sublime Text 3\footnote{https://www.sublimetext.com/3}, é um editor de texto bastante utilizados por programadores, por possuir apoio para diversas linguagens de programação, incluindo textos em LaTeX.
	% subsection sublime_text_3 (end)

	

% subsection subsection_name (end)
% section engenharia_de_software (end)