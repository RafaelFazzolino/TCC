\chapter[Proposta]{Proposta}

Este capítulo apresenta detalhes sobre a pesquisa realizada durante a primeira e segunda etapa deste trabalho de conclusão de curso. O capítulo foi organizado nas seções de \textit{Obtenção da base teórica}, a partir da utilização de revisão sistemática, e \textit{Adaptação e implementação}, onde serão especificados os objetivos práticos da pesquisa.

\section{Obtenção da base teórica} % (fold)
\label{sec:obtenção_da_base_teórica}
	
	Como qualquer trabalho científico, para alcançar os objetivos da pesquisa, faz-se necessária a realização de uma pesquisa bibliográfica buscando obter uma base teórica suficiente para sustentar o trabalho realizado. Durante este trabalho de conclusão de curso, como uma forma de obtenção da base teórica, será utilizada a técnica de revisão sistemática.

	O grande foco da revisão sistemática deverá ser as técnicas de resolução do problema de SLAM utilizadas atualmente, em diferentes contextos. Levando sempre em consideração a utilização das mesmas em um contexto simplificado e educacional. A identificação destas técnicas possibilita a seleção e adaptação das mesmas para viabilizar sua implementação no contexto simplificado.
% section obtenção_da_base_teórica (end)

\section{Adaptação e implementação} % (fold)
\label{sec:adaptação_e_implementação}

	Após a identificação e análise de diferentes técnicas de resolução do problema de SLAM, nos mais variados contextos, o trabalho selecionará as mais viáveis para realizar uma adaptação ao contexto simplificado da robótica educacional. 
	Com a seleção das mesmas, integrações entre técnicas poderão ser utilizadas para maximizar a qualidade da solução.

	O contexto que será aplicado envolve o kit de robótica Mindstorms, da Lego, e a utilização dos seguintes sensores:

	\begin{itemize}
		\item sonar,
		\item odometria e, possivelmente,
		\item bússola.
	\end{itemize}

	Ou seja, as técnicas obtidas, se utilizarem sensores diferentes dos listados acima, deverão ser adaptadas buscando a implementação da mesma com apenas estes sensores.
	
% section adaptação_e_implementação (end)