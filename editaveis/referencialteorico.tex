%\part{Aspectos Gerais}

\chapter[Referencial Teórico]{Referencial Teórico}
	
	Durante esta seção, questões referentes a todo o contexto abordado neste trabalho serão apresentadas e descritas de forma prática para facilitar o total entendimento do tema trabalhado.

\section{A Robótia e a Auto-Localização}

O nascimento da robótica se deu no contexto industrial, onde a automação de atividades repetitivas garantiu maior eficiência e, consequentemente maior lucro \cite{roboticaIndustrial}. Porém, com o passar dos anos, a robótica vem se expandindo e fazendo parte da vida cotidiana de muitos \cite{teachingWithRoboticKit}. A robótica é uma importante ferramenta que apóia o trabalho humano em diversos contextos, seja para a limpeza de uma casa \cite{melhoramentoServicoLimpeza} ou até para explorar novos planetas, por exemplo \cite{explore_marte}. 

Segundo \cite{roboticaIndustrial}, a palavra \textit{automação} traz à mente a noção de que a máquina será capaz de sentir e interagir com o ambiente, conseguindo se localizar e navegar por ele, executando suas atividades. Para que esta navegação seja possível, o robô precisa obter informações sobre o ambiente. Estas informações são obtidas a partir da utilização de sensores \cite{simpleMobile}. Segundo \cite{agenteExploratorioKalman}, existem inúmeros tipos de sensores, desde sensores de toque até sensores de visão ou de som. Os sensores mais utilizados em robôs móveis, segundo \cite{agenteExploratorioKalman}, são:
\begin{itemize}
	\item \textit{Odômetro}:

		São sensores de implementação simples e de baixo custo. Este tipo de sensor conta a quantidade de rotações de cada roda do robô, o que permite calcular o trajeto percorrido pelo mesmo. Esta técnica é conhecida como \textit{dead-reckoning}, como é apresentado por \cite{integrationVisionSLAMnonlinear}. Segundo \cite{agenteExploratorioKalman}, técnicas como o \textit{dead-reckoning} são bastante suscetíveis a erros, graças ao não alinhamento das rodas, derrapagens das mesmas e até erros no sinal dos sensores.

	\item \textit{Câmera}:

		A utilização de câmeras pode ser bastante útil quando se deseja navegar em um ambiente fechado, construído pelo homem e com características bem definidas, como afirma \cite{agenteExploratorioKalman}. Porém, sua utilização possui uma exigência computacional \cite{localizacaoEMapeamentoPaulo} que muitas vezes pode inviabilizá-la.

	\item \textit{Sonar}:

		É um tipo de sensor de proximidade barato e de fácil utilização e, por esse motivo, é bastante utilizado em robôs móveis para ambientes fechados \cite{agenteExploratorioKalman}. Ainda segundo \cite{agenteExploratorioKalman}, em ambientes abertos, este é um sensor falho, devido ao seu alcance limitado e por não ser direcionado.

	\item \textit{Infravermelho}:

		É um tipo de sensor muito semelhante aos sonares, porém estes são direcionados, ou seja, são capazes de identificar a direção do objeto \cite{theCleaningProject}.

	\item \textit{Laser}:

		Também é um tipo de sensor semelhante aos sonares e sensores de infravermelho, porém estes, além de serem direcionados, são mais precisos. O que os tornam equipamentos mais caros \cite{agenteExploratorioKalman}.

\end{itemize}

A utilização de cada tipo de sensor se dá de acordo com o contexto em que se deseja navegar, levando em consideração a maneira mais viável de entender o ambiente ao seu redor \cite{agenteExploratorioKalman}. Utilizando os sensores, o robô possuirá informações sobre o ambiente, e precisa processá-las para \textit{entender} o ambiente. A forma como serão processadas essas informações também depende do contexto da navegação, como mostra \cite{roboBulldozerIV}.

Para que a navegação ocorra sem colisões em obstáculos, o robô precisa de informações sobre o ambiente, assim como informações relacionadas a sua posição em relação a este ambiente \textbf{[REFERENCIAR]}. Para solucionar este problema, uma técnica bastante difundida é a utilização de mapas, como mostra \cite{roboBulldozerIV}. Nesta técnica, o robô recebe o mapa do ambiente que se deseja navegar a prióri e, a partir deste mapa, traça sua trajetória sem obstáculos. Utilizando este mapa e informações do ambiente, o robô é capaz de se auto-localizar no ambiente, navegando com maior precisão.

Porém, esta técnica possui requisitos que, muitas vezes, são inviáveis, como o conhecimento prévio do mapa do ambiente. Caso o objetivo seja navegar em um ambiente desconhecido, onde não há mapa nem pontos de referência já conhecidos, deve-se buscar formas de se auto-localizar e navegar utilizando apenas as informações obtidas pelos sensores. Para isso, vem sendo discutida entre toda a comunidade de pesquisadores em robótica móvel, a resolução do problema de SLAM (Auto-Localização e Mapeamento de Ambientes Simultâneos). Segundo \cite{slamProblem}, SLAM é considerado, pela comunidade, como o \textit{Santo Graal} da robótica móvel. A grande maioria dos algorítmos de auto-localização existentes aplica uma técnica de resolução do problema de SLAM \textbf{[REFERENCIAR]}.

\section{O Problema de SLAM} % (fold)
\label{sec:section_name}

% section section_name (end)


\section{Robótica Educacional} % (fold)
\label{sec:robótica_educacional}

% section robótica_educacional (end)

\section{Proposta de Adaptação} % (fold)
\label{sec:proposta_de_adaptação}

% section proposta_de_adaptação (end)