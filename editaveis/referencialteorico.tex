%\part{Aspectos Gerais}

\chapter[Referencial Teórico]{Referencial Teórico}
	
	Durante esta seção, questões referentes a todo o contexto abordado neste trabalho serão apresentadas e descritas de forma prática para facilitar o total entendimento do tema trabalhado.

\section{A Robótia e a Auto-Localização}
	
	\subsection{O Nascimento da Robótica} % (fold)
	\label{sub:o_nascimento_da_robotica}

	Grande parte da capacidade do ser humano de se adaptar ao ambiente, sobrevivendo e evoluindo constantemente se dá à utilização, desde os primórdios da humanidade, de ferramentas de auxílio em atividades importantes para o desenvolvimento de uma civilização, ou até mesmo em questões relacionadas à sobrevivência básica, como busca por alimentação e moradia. Devido ao fato do ser humano sempre buscar evolução, as ferramentas utilizadas por nós também possuem uma tendência a serem evoluídas com o tempo.

	Um exemplo simples que retrata a busca por melhoria nos instrumentos de trabalho pode ser observado em trechos descritos por Aristóteles, em meados do século IV a.c., onde o mesmo discute a possibilidade dos instrumentos realizarem suas próprias tarefas, obedecendo ou, até mesmo, antecipando o desejo das pessoas. Aristóteles ainda não sabia, mas já estava descrevendo o futuro de nossas ferramentas, o nascimento da Robótica.

	Durante os séculos seguintes a humanidade questionou o uso da ciência dentro da Indústria, para que a produção de alimentos e utensilios que possam minimizar as dificuldades encontradas durante a evolução da Humanidade possa ser evoluída e melhorada constantemente. Ao final do século XVI, Francis Bacon já discutia a ideia de que a sabedoria devesse ser aplicada na prática, ou seja, a ciência deveria ser utilizada dentro das Indústrias. Bacon afirmava, ainda, que o Homem possui o dever de se organizar com o objetivo de melhorar e transformar as condições de vida.

	Esta aplicação da ciência na indústria, descrita por Francis Bacon, passou a ser visível dois séculos depois. Quando James Watt desenvolveu, em 1769, a primeira Máquina a Vapor. A partir daí, as ferramentas humanas não necessitavam mais da força do homem para funcionarem, tortando-as muito mais autônomas, se comparado com as ferramentas existentes anteriormente. Esta fantástica evolução apresentou a toda a humanidade a enorme capacidade de evolução social e econômica, quando se tem a aplicação da Ciência nos meios Industriais.

	A partir daí, a humanidade se dedicou a utilizar a ciência para a evolução constante de suas ferramentas, alcançando em 1921, o termo \textit{"Robô"}. Este termo foi apresentado durante uma peça teatral chamada de \textit{Os Robôs Universais de Russum (R.U.R)}, a qual apresentava os robôs como sendo seres autômatos que acabam se rebelando contra os humanos. A palavra robô é derivada da palavra \textit{robota}, de origem eslava, que significa \textit{trabalho forçado}. ~\cite{roboticaIndustrial}.

	Na década de 40, o escritor Isaac Asimov popularizou o conceito de robô como sendo uma máquia de aparência humana, porem sem sentimentos. Segundo ele, os comportamentos presentes no robô seriam definidos a partir de programação realizada por seres humanos. Asimov criou o termo \textit{Robótica}, definindo-o como o estudo dos robôs, especificando, ainda, as três leis fundamentais da robótica:

	\begin{enumerate}
		\item Um robô não pode fazer mal a um ser humano e nem consentir, permanecendo inoperante,
 que um ser humano se exponha a situação de perigo; 
 		\item Um robô deve obedecer sempre às ordens de seres humanos, exceto em circunstâncias em
 que estas ordens entrem em conflito com a 1ª lei; 
 		\item Um robô deve proteger a sua própria existência, exceto em circunstâncias que entrem em
 conflito com a 1ª e 2ª leis.
	\end{enumerate}

	% subsection o_nascimento_da_robótica (end)
