%\part{Aspectos Gerais}

\chapter[Referencial Teórico]{Referencial Teórico}
	
	Durante esta seção, questões referentes a todo o contexto abordado neste trabalho serão apresentadas e descritas de forma prática para facilitar o total entendimento do tema trabalhado.

\section{A Robótia e a Auto-Localização}

O nascimento da robótica se deu no contexto industrial, onde a automação de atividades repetitivas garantiu maior eficiência e, consequentemente maior lucro \cite{roboticaIndustrial}. Porém, com o passar dos anos, a robótica vem se expandindo e fazendo parte da vida cotidiana de muitos \cite{teachingWithRoboticKit}. A robótica é uma importante ferramenta que apóia o trabalho humano em diversos contextos, seja para a limpeza de uma casa \cite{melhoramentoServicoLimpeza} ou até para explorar novos planetas, por exemplo \cite{explore_marte}.

No contexto da robótica móvel, existem por exemplo, os robôs de serviço. Os quais vêm sendo largamente evoluídos pela comunidade de robótica \cite{theCleaningProject}. Ainda segundo \cite{theCleaningProject}, estes robôs, geralmente, desempenham ações que contemplam diversas aplicações, como:

\begin{itemize}
	\item Aplicações que envolvam risco de vida significativo para humanos \cite{explore_marte};
	\item Funções economicamente desvantajosas no uso de trabalhadores humanos \cite{roboticaIndustrial};
	\item Uso humanitário (cadeiras de rodas autônomas, por exemplo);
	\item Uso educacional \cite{teachingToolsState-Art}.
\end{itemize} 

Robôs de serviço capazes de se deslocar livremente pelo ambiente, conhecendo-o, poderão realizar suas atividades de forma mais eficiente \cite{theCleaningProject}, o que garante sua autonomia.

Segundo \cite{roboticaIndustrial}, a palavra \textit{automação} traz à mente a noção de que a máquina será capaz de sentir e interagir com o ambiente, conseguindo se localizar e navegar por ele, executando suas atividades. Para que esta navegação seja possível, o robô precisa obter informações sobre o ambiente, as quais são obtidas a partir da utilização de sensores \cite{simpleMobile}. Segundo \cite{agenteExploratorioKalman}, existem inúmeros tipos de sensores, desde sensores de toque até sensores de visão ou de som. Os sensores mais utilizados em robôs móveis, segundo \cite{agenteExploratorioKalman}, são:
\begin{itemize}
	\item \textit{Odômetro}:

		São sensores de implementação simples e de baixo custo. Este tipo de sensor conta a quantidade de rotações de cada roda do robô, o que permite calcular o trajeto percorrido pelo mesmo. Esta técnica é conhecida como \textit{dead-reckoning}, como é apresentado por \cite{integrationVisionSLAMnonlinear}. Segundo \cite{agenteExploratorioKalman}, técnicas como o \textit{dead-reckoning} são bastante suscetíveis a erros, graças ao não alinhamento das rodas, derrapagens das mesmas e até erros no sinal dos sensores.

	\item \textit{Câmera}:

		A utilização de câmeras pode ser bastante útil quando se deseja navegar em um ambiente fechado, construído pelo homem e com características bem definidas, como afirma \cite{agenteExploratorioKalman}. Porém, sua utilização possui uma exigência computacional \cite{localizacaoEMapeamentoPaulo} que muitas vezes pode inviabilizá-la.

	\item \textit{Sonar}:

		É um tipo de sensor de proximidade barato e de fácil utilização e, por esse motivo, é bastante utilizado em robôs móveis para ambientes fechados \cite{agenteExploratorioKalman}. Ainda segundo \cite{agenteExploratorioKalman}, em ambientes abertos, este é um sensor falho, devido ao seu alcance limitado e por não ser direcionado.

	\item \textit{Infravermelho}:

		É um tipo de sensor muito semelhante aos sonares, porém estes são direcionados, ou seja, são capazes de identificar a direção do objeto \cite{theCleaningProject}.

	\item \textit{Laser}:

		Também é um tipo de sensor semelhante aos sonares e sensores de infravermelho, porém estes, além de serem direcionados, são mais precisos. O que os tornam equipamentos mais caros \cite{agenteExploratorioKalman}.

\end{itemize}

A utilização de cada tipo de sensor se dá de acordo com o contexto em que se deseja navegar, levando em consideração a maneira mais viável de entender o ambiente ao seu redor \cite{agenteExploratorioKalman}. Utilizando os sensores, o robô obterá informações sobre o ambiente, e precisa processá-las para \textit{entender} o mesmo. A maneira de processar (analisar) essas informações também depende do contexto da navegação, como mostra \cite{roboBulldozerIV}.

Para que a navegação ocorra sem colisões em obstáculos, o robô precisa, além de informações sobre o ambiente, informações relacionadas a sua posição em relação a este ambiente \cite{theCleaningProject}. Para solucionar este problema, uma técnica bastante difundida é a utilização de mapas, como mostra \cite{roboBulldozerIV}. Nesta técnica, o robô recebe o mapa do ambiente que se deseja navegar a prióri e, a partir deste mapa, traça sua trajetória sem obstáculos. Utilizando este mapa e informações do ambiente, o robô é capaz de se auto-localizar no ambiente, navegando com maior precisão \cite{roboBulldozerIV}.

Porém, esta técnica possui requisitos que, muitas vezes, são inviáveis, como o conhecimento prévio do mapa do ambiente. Caso o objetivo seja navegar em um ambiente desconhecido, onde não há mapa nem pontos de referência já conhecidos, deve-se buscar formas de se auto-localizar e navegar utilizando apenas as informações obtidas pelos sensores. Para isso, vem sendo discutida entre toda a comunidade de pesquisadores em robótica móvel, a resolução do problema de SLAM (Auto-Localização e Mapeamento de Ambientes Simultâneos). Segundo \cite{slamProblem}, SLAM é considerado, pela comunidade, como o \textit{Santo Graal} da robótica móvel.

%----------------------------O PROBLEMA DE SLAM-------------------------------------
\section{O Problema de SLAM} % (fold)
\label{sec:section_name}

Auto-localização e Mapeamento de Ambientes Simultâneos (SLAM) é uma técnica muito utilizada para navegação em diferentes contextos, como na navegação marítima, por exemplo \cite{slamProblem}. Segundo \cite{slamProblem}, uma máquina capaz de partir de um ponto de origem desconhecido em um ambiente desconhecido e, utilizando seus sensores, mapear o ambiente, utilizando este mapa, simultaneamente, para se auto-localizar no ambiente faz juz a palavra \textit{robô}. Desse modo, pesquisadores em robótica móvel vêm buscando soluções elegantes para o problema de SLAM \cite{integrationVisionSLAMnonlinear}.

Inúmeras soluções para o problema de SLAM, em diferentes contextos, já foram desenvolvidas e vêm sendo refinadas, com o intuito de maximizar a efetividade da navegação \cite{theCleaningProject}. Segundo \cite{circumventingAssociationSLAM}, graças a natureza imperfeita presente nos sensores, a falta de previsibilidade em ambientes reais de atuação e a necessidade de aproximações para viabilizar a análise de decisões computacionais, a robótica é uma ciência dependente de algorítmos probabilísticos. E por esse motivo, diversas  abordagens de análise probabilística podem ser seguidas \cite{circumventingAssociationSLAM}.

De acordo com \cite{slamProblem}, a abordagem mais utilizada para resolução do problema de SLAM, é a utilização do filtro de Kalman. \cite{theCleaningProject} descreve o filtro de Kalman como: \textit{"Um algorítmo recursivo de processamento de informações, proporcionando a estimação ótima do estado de um sistema dinâmico com ruído linear"}.

Este filtro utiliza diversas disciplinas, como mostra \cite{theCleaningProject}:
\begin{itemize}
	\item Mínimos quadrados;
	\item Teoria das probabilidades;
	\item Sistemas dinâmicos;
	\item Sistemas estocásticos;
	\item Álgebra.
\end{itemize}

Porém, a utilização da abordagem do filtro de Kalman, geralmente, exige requisitos computacionais de alto custo, dificultando sua utilização em um contexto de robôs simples, onde baixo processamento e pouca memória fazem parte da realidade \cite{agenteExploratorioKalman}.

 Em seu trabalho, \cite{slamProblem} exemplifica este problema, quando quantifica a relação entre o crescimento do requisito computacional necessário para a quantidade de pontos de referência no ambiente, com a utilização do filtro de Kalman. Segundo ele, enquanto a quantidade de pontos de referência aumenta em N, os requisitos computacionais e de armazenamento necessários aumentam em N\textsuperscript{2}. \cite{slamProblem} mostra que este problema pode ser solucionado, em parte, utilizando técnicas de aproximação delimitada, por exemplo. Entretando, estas técnicas minimizam o problema, mas não o solucionam completamente \cite{slamProblem}.

Outra abordagem bastante utilizada para resoluçao do problema de SLAM envolve a utilização do filtro de Partículas, que é uma técnica para implementação de um \textit{Filtro Bayesiano} de forma recursiva utilizando o método de \textit{Monte Carlo} \cite{integrationVisionSLAMnonlinear}. O método de Monte Carlo baseia-se em amostragens aleatórias em grande quantidade com o objetivo de estabelecer o valor de uma grandeza que não é disponível através de uma expressão matemática \cite{mooney1997monte} e \cite{comparacaoKalmanParticulas}.

O funcionamento desta abordagem se baseia em subdividir o ambiente em partículas espalhas uniformemente, que representam o robô munido de seus sensores \cite{comparacaoKalmanParticulas}. Cada partícula é uma hipótese da posição atual do robô \cite{dp-slam}. O robô e as partículas obtém informações do ambiente, o robô utilizando seus sensores e as partículas utilizando equações matemáticas para geração destes dados \cite{comparacaoKalmanParticulas}. As informações obtidas pelo robô real são comparadas com as informações de cada partícula, excluindo as partículas que não oferecem informações semelhantes às do robô real \cite{comparacaoKalmanParticulas}. Com o passar dos ciclos, as partículas que ainda continuam no ambiente representam a posição atual do robô \cite{comparacaoKalmanParticulas}.

A partir da comparação entre estas duas abordagens (filtro de Kalman e filtro de Partículas) \cite{comparacaoKalmanParticulas}, observou-se que as duas possuem vantagens e desvantagens. O filtro de Kalman converge mesmo com um estado inicial impreciso, fornecendo informações sobre as incertezas presentes em cada estágio e permitindo, ainda, a incorporação de toda a informação disponível (sensores) para minimizar a mergem de erro da estimativa \cite{comparacaoKalmanParticulas}. Porém, ele só garante a efetividade da estimativa para sistemas lineares com ruídos gaussianos \cite{comparacaoKalmanParticulas}. O mesmo não resolve o problema do sequestro do robô.

Já o filtro de Partículas permite sua utilização em sistemas sem ruído gaussiano, e soluciona o problema do sequestro do robô \cite{filtroParticulasComLEGO}. Entre suas desvantagens se encontra o alto custo computacional e a dificuldade da definição da quantidade ideal de partículas a serem utilizadas \cite{comparacaoKalmanParticulas}.

O sequestro do robô, citado acima, é um problema especial de localização global no campo da robótica móvel, onde o desafio deste problema, segundo \cite{sequestroRobo}, é fazer com que o robô seja capaz de se localizar em um mapa após ser sequestrado (ser retirado de sua posição e colocado em uma posição desconhecida no mapa).

As duas grandes abordagens apresentadas acima são amplamente utilizadas em diversos contextos do mundo real, contemplando questões importantes a serem estudadas e refinadas por estudiosos de engenharia de todo o mundo, desde estudantes que buscam ingressar nesta área até profissionais da área \textbf{REFERENCIAR}. Desse modo, vê-se a necessidade da adaptaçao de técnicas de resolução do problema de SLAM, seja a partir de uma abordagem baseada no filtro de Kalman, ou de uma abordagem baseada no filtro de Partículas, para um contexto Educacional. Onde os sensores, capacidades de processamento e memória são bastante limitados \textbf{REFERENCIAR}.
% section section_name (end)

%---------------------------------ROBÓTICA EDUCACIONAL------------------------------------
\section{Robótica Educacional} % (fold)
\label{sec:robótica_educacional}

A utilização da tecnologia como ferramenta de apoio ao aprendizado em escolas e universidades vem sendo ampliada com o passar dos anos \textbf{REFERENCIAR}. Principalmente após o reconhcimento dos benefícios da utilização destas tecnologias como abordagem de ensino, onde os alunos são inseridos no problema real e instigados a solucioná-lo \textbf{REFERENCIAR}.

Entre os benefícios apresentados por \textbf{REFERENCIAR, REFERENCIAR..}, pode-se destacar, o maior interesse dos alunos sobre o tema, a abordagem facilitadora para o relacionamento entre aluno, professor e conteúdo \cite{roboticaEducacionalAulasMatematica}, a experiência em trabalho em grupo, a multidisciplinariedade e a construção do conhecimento por parte do aluno.

Referente ao aperfeiçoamento do interesse dos alunos sobre o tema, \cite{construcionismoPapert} e \textbf{REFERENCIAR}, apresentam motivos deste aperfeiçoamento, como:
\begin{itemize}
	\item \textit{Curiosidade sobre a tecnologia}:

		Alunos que estão acostumados a aprender a partir do modo tradicional de ensino, quando inseridos em um contexto de trabalho real utilizando uma tecnologia nova, sentem curiosidade sobre a tecnologia, que foge ao padrão \textit{caneta/caderno/livro} \cite{construcionismoPapert}. Desse modo, a atenção dos alunos já está garantida, o que muitas vezes é uma atividade difícil de se obter na abordagem tradicional de ensino \textbf{REFERENCIAR}.

	\item Reconhecimento do problema real como um problema cotidiano do aluno:

		De acordo com \textbf{REFERENCIAR}, em situações em que o aluno reconhece o problema estudado como um problema cotidiano de sua vida, seu interesse é ampliado de maneira extraordinária. A partir deste interesse, o aluno busca compreender os detalhes do problema e solucioná-lo da melhor maneira possível, realizando adaptações de contexto, quando necessárias \textbf{REFERENCIAR}. Esta adaptação envolve diversos critérios de aprendizado que são desenvolvidos pelo próprio aluno, sem a necessidade do ensino propriamente dito \textbf{REFERENCIAR}.

	\item Utilização da abordagem \textit{Hands-on}:

		A abordagem \textit{Hands-on}, segundo \textbf{REFERENCIAR}, sugere a construção do conhecimento a partir do desenvolvimento de uma solução prática por parte do aluno. Esta abordagem incorpora a solução de um problema específico, onde a mesma envolve a utilização/conhecimento de diversas áreas de conhecimento, como matemática, física, engenharia e lógica \textbf{REFERENCIAR}. Desse modo, esta abordagem garante o aprendizado dos conteúdos básicos de modo a contornar a abordagem tradicional de ensino, maximizando o interesse dos alunos participantes \textbf{REFERENCIAR}.
\end{itemize}

Sobre o aperfeiçoamento da relação entre alunos, professor e conteúdo, \textbf{REFERENCIAR}, destaca a reformulação do padrão de ensino e aprendizagem durante a aula. De acordo com \textbf{REFERENCIAR}, no contexto do ensino tradicional, o aluno busca \textit{clonar} o conhecimento do professor, decorando informações para, no futuro, utilizá-los no contexto real. Entretando, segundo \textbf{REFERENCIAR}, esta abordagem, além de limitar o aprendizado do aluno ao conhecimento do professor, minimiza a capacidade de aprendizado do aluno, já que o conhecimento não é construído, e sim repassado.

Já na abordagem da utilização de tecnologias no contexto educacional, a resolução prática do problema atacado faz com que alunos e professores trabalhem lado-a-lado, muitas vezes realizando troca de papéis aluno/professor \textbf{REFERENCIAR}. Desse modo, em muitas ocasiões, os alunos se encontram explicando uma possível solução do problema ao professor, o que acaba com o aprendizado limitado ao conhecimento do professor \textbf{REFERENCIAR}. \textbf{REFERENCIAR} descreve o professor como um fio condutor do conhecimento, e não a fonte do mesmo\textbf{REFERENCIAR}.

Já, de acordo com a relação da experiência do aluno em trabahos em grupo, a abordagem \textit{Hands-on} garante que todos os envolvidos na solução devem interagir entre si, trocando conhecimento e ideias \textbf{REFERENCIAR}. Qualquer contexto em que se deseja trabalhar, atualmente, envolve inúmeras atividades de trabalho em grupo \textbf{REFERENCIAR}. Desse modo, o aperfeiçoamento da capacidade de trabalhar em grupo é uma atividade essencial para o futuro profissional do aluno \textbf{REFERENCIAR}.

Outra característica importante da utilização da tecnologia como ferramenta de aprendizado, é a multidisciplinariedade \textbf{REFERENCIAR}, já que conteúdos referentes a diversas disciplinas são trabalhados e adaptados para buscar a solução do problema atacado. \textbf{REFERENCIAR} sugere a utilização de atividades multidisciplinares, o que maximiza o conhecimento global dos conteúdos, contornando o problema do conhecimento parcial.

Além de todos os benefícios apresentados acima, uma grande qualidade da abordagem \textit{Hands-on} é referente a construção do conhecimento por parte do aluno \textbf{REFERENCIAR}. Esta construção é gerada utilizando conhecimentos básicos de diferentes disciplinas e segue uma filosofia construcionista, como é apresentado por \cite{construcionismoPapert}.
 
\subsection{Construcionismo} % (fold)
\label{sub:construcionismo}

	De acordo com \cite{construcionismoPapert}, quando o aluno se sente imerso no problema trabalhado, a maneira com que o mesmo aprende é aperfeiçoada, maximizando as relações entre aluno, professor e conteúdo. Este pensamento segue uma filosofia construcionista. Esta filosofia, de acordo com \cite{construcionismoPapert}, implica no objetivo de ensinar, de forma a produzir o máximo de aprendizagem a partir do mínimo de ensino.

	A utilização de aplicações rotineiras do mundo real é uma forma bastante eficaz de obter máximo interesse dos alunos nos conteúdos apresentados \cite{construcionismoPapert}. Desse modo, inserir alunos na resolução de problemas presentes no contexto da robótica móvel, por exemplo, fará com que os mesmos aprendam, com eficiência, diversos temas recorrentes no contexto mundial da robótica móvel \textbf{REFERENCIAR}, além de conteúdos presentes em diversas disciplinas, como matemática e física \cite{roboticaEducacionalAulasMatematica}.

	O construcionismo é baseado em uma filosofia construtivista \cite{construcionismoPapert}, que afirma que o conhecimento não deve ser uma cópia da realidade, e sim uma construção, realizada pelo aluno a partir da sua interação com o ambiente do problema \cite{oQueEConstrutivismo}.

% subsection construcionismo (end)

Segundo \cite{simpleRobotsIntroductionEng}, a utilização da robótica no meio da educação traz inúmeros benefícios que vão alem dos objetivos diretos da melhoria da aprendizagem. De acordo com ele, sua utilização garante a introdução dos estudantes nos problemas recorrentes do contexto de Engenharia mundial. Esta introdução possui grande importância para a evolução do país, já que a demanda por profissionais qualificados na área de Engenharia, no mundo atual, possui um crescimento ascendente \cite{simpleRobotsIntroductionEng}.

\subsection{Mindstorm NXT} % (fold)
\label{sub:kit_mindstorm}

% subsection kit_mindstorm (end)
% section robótica_educacional (end)
